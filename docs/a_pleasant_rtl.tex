
%% bare_conf_compsoc.tex
%% V1.4b
%% 2015/08/26
%% by Michael Shell
%% See:
%% http://www.michaelshell.org/
%% for current contact information.
%%
%% This is a skeleton file demonstrating the use of IEEEtran.cls
%% (requires IEEEtran.cls version 1.8b or later) with an IEEE Computer
%% Society conference paper.
%%
%% Support sites:
%% http://www.michaelshell.org/tex/ieeetran/
%% http://www.ctan.org/pkg/ieeetran
%% and
%% http://www.ieee.org/

%%*************************************************************************
%% Legal Notice:
%% This code is offered as-is without any warranty either expressed or
%% implied; without even the implied warranty of MERCHANTABILITY or
%% FITNESS FOR A PARTICULAR PURPOSE!
%% User assumes all risk.
%% In no event shall the IEEE or any contributor to this code be liable for
%% any damages or losses, including, but not limited to, incidental,
%% consequential, or any other damages, resulting from the use or misuse
%% of any information contained here.
%%
%% All comments are the opinions of their respective authors and are not
%% necessarily endorsed by the IEEE.
%%
%% This work is distributed under the LaTeX Project Public License (LPPL)
%% ( http://www.latex-project.org/ ) version 1.3, and may be freely used,
%% distributed and modified. A copy of the LPPL, version 1.3, is included
%% in the base LaTeX documentation of all distributions of LaTeX released
%% 2003/12/01 or later.
%% Retain all contribution notices and credits.
%% ** Modified files should be clearly indicated as such, including  **
%% ** renaming them and changing author support contact information. **
%%*************************************************************************


% *** Authors should verify (and, if needed, correct) their LaTeX system  ***
% *** with the testflow diagnostic prior to trusting their LaTeX platform ***
% *** with production work. The IEEE's font choices and paper sizes can   ***
% *** trigger bugs that do not appear when using other class files.       ***                          ***
% The testflow support page is at:
% http://www.michaelshell.org/tex/testflow/



\documentclass[conference,compsoc]{IEEEtran}
% Some/most Computer Society conferences require the compsoc mode option,
% but others may want the standard conference format.
%
% If IEEEtran.cls has not been installed into the LaTeX system files,
% manually specify the path to it like:
% \documentclass[conference,compsoc]{../sty/IEEEtran}





% Some very useful LaTeX packages include:
% (uncomment the ones you want to load)


% *** MISC UTILITY PACKAGES ***
%
%\usepackage{ifpdf}
% Heiko Oberdiek's ifpdf.sty is very useful if you need conditional
% compilation based on whether the output is pdf or dvi.
% usage:
% \ifpdf
%   % pdf code
% \else
%   % dvi code
% \fi
% The latest version of ifpdf.sty can be obtained from:
% http://www.ctan.org/pkg/ifpdf
% Also, note that IEEEtran.cls V1.7 and later provides a builtin
% \ifCLASSINFOpdf conditional that works the same way.
% When switching from latex to pdflatex and vice-versa, the compiler may
% have to be run twice to clear warning/error messages.






% *** CITATION PACKAGES ***
%
\ifCLASSOPTIONcompsoc
  % IEEE Computer Society needs nocompress option
  % requires cite.sty v4.0 or later (November 2003)
  \usepackage[nocompress]{cite}
\else
  % normal IEEE
  \usepackage{cite}
\fi
% cite.sty was written by Donald Arseneau
% V1.6 and later of IEEEtran pre-defines the format of the cite.sty package
% \cite{} output to follow that of the IEEE. Loading the cite package will
% result in citation numbers being automatically sorted and properly
% "compressed/ranged". e.g., [1], [9], [2], [7], [5], [6] without using
% cite.sty will become [1], [2], [5]--[7], [9] using cite.sty. cite.sty's
% \cite will automatically add leading space, if needed. Use cite.sty's
% noadjust option (cite.sty V3.8 and later) if you want to turn this off
% such as if a citation ever needs to be enclosed in parenthesis.
% cite.sty is already installed on most LaTeX systems. Be sure and use
% version 5.0 (2009-03-20) and later if using hyperref.sty.
% The latest version can be obtained at:
% http://www.ctan.org/pkg/cite
% The documentation is contained in the cite.sty file itself.
%
% Note that some packages require special options to format as the Computer
% Society requires. In particular, Computer Society  papers do not use
% compressed citation ranges as is done in typical IEEE papers
% (e.g., [1]-[4]). Instead, they list every citation separately in order
% (e.g., [1], [2], [3], [4]). To get the latter we need to load the cite
% package with the nocompress option which is supported by cite.sty v4.0
% and later.





% *** GRAPHICS RELATED PACKAGES ***
%
\ifCLASSINFOpdf
  % \usepackage[pdftex]{graphicx}
  % declare the path(s) where your graphic files are
  % \graphicspath{{../pdf/}{../jpeg/}}
  % and their extensions so you won't have to specify these with
  % every instance of \includegraphics
  % \DeclareGraphicsExtensions{.pdf,.jpeg,.png}
\else
  % or other class option (dvipsone, dvipdf, if not using dvips). graphicx
  % will default to the driver specified in the system graphics.cfg if no
  % driver is specified.
  % \usepackage[dvips]{graphicx}
  % declare the path(s) where your graphic files are
  % \graphicspath{{../eps/}}
  % and their extensions so you won't have to specify these with
  % every instance of \includegraphics
  % \DeclareGraphicsExtensions{.eps}
\fi
% graphicx was written by David Carlisle and Sebastian Rahtz. It is
% required if you want graphics, photos, etc. graphicx.sty is already
% installed on most LaTeX systems. The latest version and documentation
% can be obtained at:
% http://www.ctan.org/pkg/graphicx
% Another good source of documentation is "Using Imported Graphics in
% LaTeX2e" by Keith Reckdahl which can be found at:
% http://www.ctan.org/pkg/epslatex
%
% latex, and pdflatex in dvi mode, support graphics in encapsulated
% postscript (.eps) format. pdflatex in pdf mode supports graphics
% in .pdf, .jpeg, .png and .mps (metapost) formats. Users should ensure
% that all non-photo figures use a vector format (.eps, .pdf, .mps) and
% not a bitmapped formats (.jpeg, .png). The IEEE frowns on bitmapped formats
% which can result in "jaggedy"/blurry rendering of lines and letters as
% well as large increases in file sizes.
%
% You can find documentation about the pdfTeX application at:
% http://www.tug.org/applications/pdftex





% *** MATH PACKAGES ***
%
%\usepackage{amsmath}
% A popular package from the American Mathematical Society that provides
% many useful and powerful commands for dealing with mathematics.
%
% Note that the amsmath package sets \interdisplaylinepenalty to 10000
% thus preventing page breaks from occurring within multiline equations. Use:
%\interdisplaylinepenalty=2500
% after loading amsmath to restore such page breaks as IEEEtran.cls normally
% does. amsmath.sty is already installed on most LaTeX systems. The latest
% version and documentation can be obtained at:
% http://www.ctan.org/pkg/amsmath





% *** SPECIALIZED LIST PACKAGES ***
%
%\usepackage{algorithmic}
% algorithmic.sty was written by Peter Williams and Rogerio Brito.
% This package provides an algorithmic environment fo describing algorithms.
% You can use the algorithmic environment in-text or within a figure
% environment to provide for a floating algorithm. Do NOT use the algorithm
% floating environment provided by algorithm.sty (by the same authors) or
% algorithm2e.sty (by Christophe Fiorio) as the IEEE does not use dedicated
% algorithm float types and packages that provide these will not provide
% correct IEEE style captions. The latest version and documentation of
% algorithmic.sty can be obtained at:
% http://www.ctan.org/pkg/algorithms
% Also of interest may be the (relatively newer and more customizable)
% algorithmicx.sty package by Szasz Janos:
% http://www.ctan.org/pkg/algorithmicx




% *** ALIGNMENT PACKAGES ***
%
%\usepackage{array}
% Frank Mittelbach's and David Carlisle's array.sty patches and improves
% the standard LaTeX2e array and tabular environments to provide better
% appearance and additional user controls. As the default LaTeX2e table
% generation code is lacking to the point of almost being broken with
% respect to the quality of the end results, all users are strongly
% advised to use an enhanced (at the very least that provided by array.sty)
% set of table tools. array.sty is already installed on most systems. The
% latest version and documentation can be obtained at:
% http://www.ctan.org/pkg/array


% IEEEtran contains the IEEEeqnarray family of commands that can be used to
% generate multiline equations as well as matrices, tables, etc., of high
% quality.




% *** SUBFIGURE PACKAGES ***
%\ifCLASSOPTIONcompsoc
%  \usepackage[caption=false,font=footnotesize,labelfont=sf,textfont=sf]{subfig}
%\else
%  \usepackage[caption=false,font=footnotesize]{subfig}
%\fi
% subfig.sty, written by Steven Douglas Cochran, is the modern replacement
% for subfigure.sty, the latter of which is no longer maintained and is
% incompatible with some LaTeX packages including fixltx2e. However,
% subfig.sty requires and automatically loads Axel Sommerfeldt's caption.sty
% which will override IEEEtran.cls' handling of captions and this will result
% in non-IEEE style figure/table captions. To prevent this problem, be sure
% and invoke subfig.sty's "caption=false" package option (available since
% subfig.sty version 1.3, 2005/06/28) as this is will preserve IEEEtran.cls
% handling of captions.
% Note that the Computer Society format requires a sans serif font rather
% than the serif font used in traditional IEEE formatting and thus the need
% to invoke different subfig.sty package options depending on whether
% compsoc mode has been enabled.
%
% The latest version and documentation of subfig.sty can be obtained at:
% http://www.ctan.org/pkg/subfig




% *** FLOAT PACKAGES ***
%
%\usepackage{fixltx2e}
% fixltx2e, the successor to the earlier fix2col.sty, was written by
% Frank Mittelbach and David Carlisle. This package corrects a few problems
% in the LaTeX2e kernel, the most notable of which is that in current
% LaTeX2e releases, the ordering of single and double column floats is not
% guaranteed to be preserved. Thus, an unpatched LaTeX2e can allow a
% single column figure to be placed prior to an earlier double column
% figure.
% Be aware that LaTeX2e kernels dated 2015 and later have fixltx2e.sty's
% corrections already built into the system in which case a warning will
% be issued if an attempt is made to load fixltx2e.sty as it is no longer
% needed.
% The latest version and documentation can be found at:
% http://www.ctan.org/pkg/fixltx2e


%\usepackage{stfloats}
% stfloats.sty was written by Sigitas Tolusis. This package gives LaTeX2e
% the ability to do double column floats at the bottom of the page as well
% as the top. (e.g., "\begin{figure*}[!b]" is not normally possible in
% LaTeX2e). It also provides a command:
%\fnbelowfloat
% to enable the placement of footnotes below bottom floats (the standard
% LaTeX2e kernel puts them above bottom floats). This is an invasive package
% which rewrites many portions of the LaTeX2e float routines. It may not work
% with other packages that modify the LaTeX2e float routines. The latest
% version and documentation can be obtained at:
% http://www.ctan.org/pkg/stfloats
% Do not use the stfloats baselinefloat ability as the IEEE does not allow
% \baselineskip to stretch. Authors submitting work to the IEEE should note
% that the IEEE rarely uses double column equations and that authors should try
% to avoid such use. Do not be tempted to use the cuted.sty or midfloat.sty
% packages (also by Sigitas Tolusis) as the IEEE does not format its papers in
% such ways.
% Do not attempt to use stfloats with fixltx2e as they are incompatible.
% Instead, use Morten Hogholm'a dblfloatfix which combines the features
% of both fixltx2e and stfloats:
%
% \usepackage{dblfloatfix}
% The latest version can be found at:
% http://www.ctan.org/pkg/dblfloatfix




% *** PDF, URL AND HYPERLINK PACKAGES ***
%
%\usepackage{url}
% url.sty was written by Donald Arseneau. It provides better support for
% handling and breaking URLs. url.sty is already installed on most LaTeX
% systems. The latest version and documentation can be obtained at:
% http://www.ctan.org/pkg/url
% Basically, \url{my_url_here}.




% *** Do not adjust lengths that control margins, column widths, etc. ***
% *** Do not use packages that alter fonts (such as pslatex).         ***
% There should be no need to do such things with IEEEtran.cls V1.6 and later.
% (Unless specifically asked to do so by the journal or conference you plan
% to submit to, of course. )


% We need listings, and need to set them up properly
% TODO: this needs to be reviewed!!
\usepackage{listings}
\lstset{
  frame=single,
  language=Verilog,
}
\makeatletter
\def\lst@makecaption{%
  \def\@captype{table}%
  \@makecaption
}
\makeatother



% correct bad hyphenation here
\hyphenation{op-tical net-works semi-conduc-tor}


\begin{document}
%
% paper title
% Titles are generally capitalized except for words such as a, an, and, as,
% at, but, by, for, in, nor, of, on, or, the, to and up, which are usually
% not capitalized unless they are the first or last word of the title.
% Linebreaks \\ can be used within to get better formatting as desired.
% Do not put math or special symbols in the title.
\title{There must be a better way\\What's wrong with Verilog and VHDL and how to fix it?}


% author names and affiliations
% use a multiple column layout for up to three different
% affiliations
\author{\IEEEauthorblockN{Andras Tantos}
\IEEEauthorblockA{SpaceX\\
Email: andras@tantosonline.com}
}

% conference papers do not typically use \thanks and this command
% is locked out in conference mode. If really needed, such as for
% the acknowledgment of grants, issue a \IEEEoverridecommandlockouts
% after \documentclass

% for over three affiliations, or if they all won't fit within the width
% of the page (and note that there is less available width in this regard for
% compsoc conferences compared to traditional conferences), use this
% alternative format:
%
%\author{\IEEEauthorblockN{Michael Shell\IEEEauthorrefmark{1},
%Homer Simpson\IEEEauthorrefmark{2},
%James Kirk\IEEEauthorrefmark{3},
%Montgomery Scott\IEEEauthorrefmark{3} and
%Eldon Tyrell\IEEEauthorrefmark{4}}
%\IEEEauthorblockA{\IEEEauthorrefmark{1}School of Electrical and Computer Engineering\\
%Georgia Institute of Technology,
%Atlanta, Georgia 30332--0250\\ Email: see http://www.michaelshell.org/contact.html}
%\IEEEauthorblockA{\IEEEauthorrefmark{2}Twentieth Century Fox, Springfield, USA\\
%Email: homer@thesimpsons.com}
%\IEEEauthorblockA{\IEEEauthorrefmark{3}Starfleet Academy, San Francisco, California 96678-2391\\
%Telephone: (800) 555--1212, Fax: (888) 555--1212}
%\IEEEauthorblockA{\IEEEauthorrefmark{4}Tyrell Inc., 123 Replicant Street, Los Angeles, California 90210--4321}}




% use for special paper notices
%\IEEEspecialpapernotice{(Invited Paper)}




% make the title area
\maketitle

% As a general rule, do not put math, special symbols or citations
% in the abstract
\begin{abstract}
Verilog and VHDL are the most popular hardware description languages (HDLs) and have been for years.
Yet, there are several problems with both that impedes effective use of these languages. It is a small
miracle that the industry managed to use these tools for such complex projects that modern chip design
requires. In this article we will list key problems with existing HDL languages, then present an
attempt at solving them. The presented solution is 'Silicon', a Python library for HDL development,
but many of the ideas we enumerate can be implemented in other frameworks, in some cases as extensions
to the existing languages as well. We will show through examples how these ideas not only help with
creating complex designs but also foster a more healthy design community, the creation
of widely reusable component libraries, true design re-use, improve code quality - especially
readability - and shorten design cycles.
\end{abstract}

% no keywords




% For peer review papers, you can put extra information on the cover
% page as needed:
% \ifCLASSOPTIONpeerreview
% \begin{center} \bfseries EDICS Category: 3-BBND \end{center}
% \fi
%
% For peerreview papers, this IEEEtran command inserts a page break and
% creates the second title. It will be ignored for other modes.
\IEEEpeerreviewmaketitle



\section{Introduction}
% no \IEEEPARstart
Many Hardware Description Languages (HDL)s have been created in the past. The most popular ones,
with the greatest following are Verilog and VHDL.

Verilog was created by Prabhu Goel, Phil Moorby, Chi-Lai Huang, and Douglas Warmke at the tail
end of 1983. Originally intended as a tool purely for simulations, a subset of the language –
the syntesizable subset - was later adopted for synthesis purposes. Since then the language
has been standardized as IEEE 1364. Many versions of it have been released, the latest in 2005.
SystemVerilog, initially an off-shoot of Verilog 2005 got integrated back into mainline Verilog
during the 2009 revision of the standard. Now Verilog is part of the SystemVerilog language,
which is standardized as IEEE 1800.

IMPORTANT!!!!
Verilog has two types of assignments: continuous and procedural. The latter is the @always thingy.

At around the same time, VHDL was created to support the needs of the U.S. Department of Defense.
It also became an IEEE standard (1076), with the latest version released in 2019.

Interestingly neither of these languages were originally intended as VLSI design tools. Verilog,
as stated before started its life as a simulation only language, while VHDL was intended as a
documentation tool.

Of course a lot has changed since then, and both of these languages are extensively used as design
tools for digital logic.

While SytemVerilog and VHDL are extremely rich languages, their synthesizable subset is much smaller.
While this subset to our knowledge is not formally defined for either languages, we will restrict
our discussion to those constructs generally supported by synthesis tools. We do that because our
main focus is on aiding the design process on not on verification.

Other languages, such as BlueSpec or Chisel are more modern attempts at HDLs. Many entities
developed and shipped various C or C-like languages to target high-level synthesis.

There are many attempts at using existing programming languages, such as Python (PyMTL, MyHDL)
or Scala (Chisel, SpinalHDL) for the development of synthesizable digital circuitry.

With all of these efforts however, the vast majority of the actual production chip development
is happening in the traditional languages: VHDL or Verilog. None of the previously mentioned
alternatives managed to break out of the academic or small industry groups where they were developed.

In the following, we will present a novel approach to RTL development. We will present a set of
arguments, why a new tool is needed, where do both Verilog and VHDL fall short. We will argue
that these languages ignore some very basic language development strategies, prevalent
and successful in the SW industry. We will show how the closed language ecosystem, the slow
evolution of the tools combined with unfortunate aspects of the languages result in a pronounced
lack of standardized component libraries, another major feature of a successful SW development
environment.

In the second half of the article, we will present a prototype Python-based implementation that
attempts to address these shortcomings. We show how taking the principles of composability, a
powerful type system and a fully functional meta-programming environment results in an RTL
development paradigm that is much more conducive to modularization and reuse. We will show
several examples of commonly performed design tasks that can easily be expressed in this new
paradigm without specific language support.

\section{What's wrong with the V languages?}

Both Verilog and VHDL is based on a dual-concept for describing a digital circuit. The high
level decomposition of a problem involves braking up the design into modules and then
subsequently smaller, hierarchical sub-modules. Once these modules are created, they
are instantiated in their parent module and 'wired' together by connecting their ports
together. This is called 'structural' decomposition and illustrated in
Listing~\ref{lst1:structural_verilog}.

\begin{lstlisting}[
  caption=Structural decomposition in Verilog,
  label=lst1:structural_verilog,
]
module fulladd (
  input logic A,
  input logic B,
  input logic Cin,
  output logic Cout,
  output logic Sum
);
  logic X1, A1, A2, A3, O1;

  xor XOR1(X1, A, B);
  xor XOR2(Sum, X1, Cin);
  and AND1(A1, A, B);
  and AND2(A2, A, Cin);
  and AND3(A3, B, Cin);
  or OR1(O1, A1, A2);
  or OR2(Cout, O1, A3);
endmodule
\end{lstlisting}

This structural description is easily expendable, but really hard to follow. The
flowing of data is not explicitly represented, because the inputs and outputs are mixed
together in the instance port-maps. Not only all wires entering and exiting a module need
to be mentioned in the same lexical block of code - which precludes a coding style where
logical operations are grouped together, even the direction of data-flow is implicit:
from looking at the instantiation site, it's not possible to tell what is an input
and what is an output.

If this was the only way to describe logic in these languages, it would be a painful
experience indeed. Thankfully, it's not. A different way of describing logic is to use continuous
assignments and procedural blocks. The code can make use of logical expressions and sequential statements
as shown in Listing~\ref{lst1:behavioral_verilog}.

\begin{lstlisting}[
  caption=Behavioral Verilog,
  label=lst1:behavioral_verilog,
]
module registered_fulladd (
  input logic clk,
  input logic A,
  input logic B,
  input logic Cin,
  output logic Cout,
  output logic Sum
);
  logic s, c;

  assign s = A ^ B ^ Cin;
  assign c = (A & B) | (A & Cin) |
             (B & Cin);
  always @posedge(clk) begin
    Sum <= s;
    Cout <= c;
  end;
endmodule
\end{lstlisting}

This behavioral description is much easier on the reader: the flow of the data is explicit and
easy to follow. Sequential statements are clearly separated from combinational logic. Not
surprisingly almost all code written in Verilog (or VHDL) follow the behavioral coding style.

The trouble is that this version of code is not composeable: once one described a module using
continuous assignments or sequential statements, it can only be instantiated in the behavioral style.

There is of course the mix style, where submodule instantiations and behavioral description are
both used within a single module. When used though, the order of statements and modules becomes
rather unintuitive due to the need of having everything declared before use and the fact that all
submodule ports need to be bound at the instantiation point.

The result of this shortcoming is that it's very hard to create generic, base-libraries. For
instance, using RAM modules, multiplexer primitives or even just registers
would be rather painful if they were shipped as module libraries. Almost no one does that of course
(with the exception of RAM modules in the form of memory compilers for ASIC targets). The way these
design primitives are expressed is an ever-increasing set of language constructs. RAMs are described
as arrays. Multiplexers as {\tt if} or {\tt case} statements. Registers as {\\t @always} blocks.
Each requiring their own new language feature, compiler support and special rules, especially as
it relates to synthesized circuitry.

A further complication is that migrating back and forth between the structural and behavioral styles
is difficult. Anyone who tried to re-target a design from FPGA (with inferred memories) to an ASIC
process (with explicit RAM instances) knows the pain.

Inserting pipeline stages into a design to achieve timing closure is also difficult. If done in the
structural style, every connection between submodules needs to be broken and new wires need to be
introduced with the appropriate registers (using {\tt @always} statements or other means) inserted
in between. If performed in the behavioral style, all nets still need to be broken in two and the
appropriate sequential statements be introduced in between. In both cases, this is a rather
error-prone process: no language-level guarantees exist to warn the coder if some wires are left
unregistered or - maybe even worse - are registered, but bypassed due to a simple search-and-replace
error. This is especially problematic in the structural coding style, where signal flow doesn't
follow code-flow, so sinks routinely appear prior to drivers. A consequence of this is that logical
errors can sneak their way into the design at a late stage of the project - during timing closure.

The type-system of the traditional languages is also severely lacking. Here though, VHDL has some
advantages. Verilog however is notorious in its lack of sophistication. Every net (in the
synthesisable subset) needs to be essentially either a single wire or an N-bit bus. While all nets
must have a predefined width, automatic type-conversions are performed between mismatched wire sizes.
This of course is convenient, because it avoids the excessive clutter of the explicit
width-conversions of VHDL, but results in many a design errors where the language
silently truncates some bits from a value.

<<<<TODO: figure out what's the story with sign- and zero-extension!>>>

When it comes to even slightly more complex types however, Verilog especially is hopelessly
inadequate. In many DSP applications for instance, fixed-fractional numbers are used. When dealing
with these values, special care must be taken during multiplication: a different subset of bits
are relevant from the result and the question of rounding needs to be addressed. Verilog doesn't
have native support for these types, which is to be expected. The problem is that it
isn't possible to create a library that provides primitives for these numeric representations:
the previously mentioned issues conspire to prevent that. While there is a way to create
unique {\tt nettype} in the language, many synthesis tools lack support for it.

More complex types, such as structs are supported in a limited way. For instance, one can create
a complex number by grouping two wires into a struct, but then one can't infer a RAM of such numbers.

<<<<TODO: check if this is true!>>>

Arrays are supported, but they are re-purposed for describing inferred memories. Thus, they are
not really a complex data structure, more a language feature.

Verilog does provide support for 'interfaces', where complex signaling between modules can be
encapsulated. This feature would be quite powerful, if not for one fatal flaw: interfaces can't
contain interfaces. While it's possible to create, for instance an interface for any of the five
sub-buses of the AXI4 interface spec, they can't at a higher level be combined into a single
AXI4 interface.

Generic (or meta-) programming is supported by both languages: it allows for describing logic
which depends on some compile-time available information. The canonical example is a module,
where the width of the input and output nets depend on some compile-time parameter.
Listing~\ref{lst1:generic_verilog} shows a simple example.

\begin{lstlisting}[
  caption=Generic Verilog,
  label=lst1:generic_verilog,
]
module register #(
  parameter WIDTH = 64,
) (
  input logic clk,
  input logic clk_en,
  input logic [WIDTH-1:0] I,
  output logic [WIDTH-1:0] O
);
  wire s, c;

  always @(posedge clk) begin
    if (clock_en == 1'b1) begin
      O <= I;
    end
  end
endmodule
\end{lstlisting}

There are more complex uses with the various 'generate' constructs, where the behavior and the
synthesized logic of the module can depend on such generic parameters. However, one pretty
quickly reaches the limits of these language features as well. For instance, while it's possible
to describe the width of a bus as a generic parameter, it's not possible to describe the type of
it. So while it's possible to create a generic adder of arbitrary width, it's not possible to
extend that to vectors or complex numbers.

While lacking these basic elements, both languages expand great effort supporting hardly ever
used constructs. For instance, they support nets with multiple drivers (and custom resolution
functions), which to our knowledge is almost never used in practice. Verilogs' {\tt tri} nets
add support for capacitively loaded nets with dynamic storage. There are edge-cases for them
for sure, but not in high-level digital logic design, which is our focus for this article.
Tri-stated nets, in-out ports are also part of the repertoire of both langauges, but their
use is rather minimal and relegated to chip-level I/O pads under almost all circumstances.

To summarize, neither language has the facilities necessary to foster a rich environment of
design re-use, the creation of reusable libraries. New features need constant revision of
the core language and the tools which is a slow process even under the best of circumstances.

Worse, since new features need constant updating of the core language and the tools, their
adoption is slow. Tool vendors are slow in coming out with new synthesizers and the conservative
design community is even slower in picking them up. Ecosystem players need to consider the lowest
common denominator, which results in new features getting adopted at a painfully slow rate,
sometimes taking a decade or longer.

\section{Solutions?}

In the following section we will introduce some ideas to overcome the shortcomings listed
previously. We will start by providing Verilog-like code-examples to introduce some concepts
but will soon start using Python-style examples later on, to talk about more involved
scenarios, especially the interplay between the various features.

\subsection{A false choice}

We believe the dichotomy between structural and behavioral descriptions is a false one.
A more useful paradigm is to treat the circuit description in the HDL as a netlist. Or, in
other words, we argue there's no need for behavioral descriptions; only a need for a sane
syntax for structural ones.

The trick is to treat function calls as if they were module instantiations: all the inputs
listed as function arguments, and all outputs as return values. For instance, instead of
the {\tt and AND1(A1, A, B);} instantiation syntax above, one could write {\tt A1 = and(A, B);}.

Logical operators can be thought of as in-fix function-calls, following any of the examples
of programming languages that support this feature. In Python for instance, {\tt A1 = A \& B}
would turn into a function (method) call {\tt A1 = A.\_\_and\_\_(B)} or {\tt A1 = B.\_\_rand\_\_(A)}
depending on which names are defined.

This single change immediately transforms structural description into a much more readable
and maintainable format. It also removes a significant use-case for behavioral descriptions.

Some module instances don't read well of course if written in a function-call format. A processor
instance would look very strange in deed if expressed as a function call. To deal with this,
we suggest that port binding should be decoupled from instance creation. Instance ports
instead could be referenced on the right-hand-side of an expression in case of outputs, creating
an output binding. Input instance ports can be referenced on the left-hand-side of an expression
to create an input binding.

Listing~\ref{lst1:mixed_inst_verilog} shows these new syntactic proposals in action. Here
we assume that logic gates have been defined as having {\tt out} as their output port and
{\tt in1} and {\tt in2} as their input ports.

\begin{lstlisting}[
  caption=Mixed instantiation syntax,
  label=lst1:mixed_inst_verilog,
]
module fulladd (
  input logic A,
  input logic B,
  input logic Cin,
  output logic Cout,
  output logic Sum
);
  logic X1, A1, A2, A3, O1;

  // Traditional instantiation
  xor XOR1(X1, A, B);
  // Create a new instance, no binding
  xor XOR2;
  // Binding ports of existing instance
  assign Sum = XOR2.out;
  assign XOR2.in1 = X1;
  assign XOR2.in2 = Cin;
  // Instantiation through function call
  assign A1 = and(A, B);
  and AND3;
  assign AND3.in1 = B;
  assign AND3.in2 = Cin;
  // Cascading instantiation
  // with operator overloads
  assign O1 = A1 | and(A, Cin));
  // Bindig of output inside expression
  assign Cout = or(O1, AND3.out);
endmodule
\end{lstlisting}

\subsection{Sequential constructs}
Verilog traditionally uses {\tt @always} blocks or their variants to describe sequential
logic. In reality though designs almost exclusively rely on {\tt @always\_ff} blocks to
create registers. Latches are discouraged by almost all design guidelines, some targets
such as FPGAs don't even have support for them.

The main question is this then: how to create registers without {\tt @always} blocks?
Our suggestion is the creation of a new module, called 'reg'. This module would have
an input and would output a registered version of the same signal. This module could then
be instantiated using the same type of constructs as described for combinational logic
as shown in Listing~\ref{lst1:reg_instantiation}.

\begin{lstlisting}[
  caption=Registers using instantiations,
  label=lst1:reg_instantiation,
]
module registered_fulladd (
  input logic clk,
  input logic A,
  input logic B,
  input logic Cin,
  output logic Cout,
  output logic Sum
);
  logic c;

  assign c = (A & B) | (A & Cin) |
             (B & Cin);
  assign Sum = reg(A ^ B ^ Cin);
  assign Cout = reg(c)
endmodule
\end{lstlisting}

There is at least one obvious problem with this construct though that needs to be addressed.
How does our reg instance from Listing~\ref{lst1:reg_instantiation} get clocked?

\subsubsection{Auto-ports}
Our proposal is the introduction of a feature, called 'auto-ports'. These ports would automatically
bind to a net in the instantiation context if they have the right name. For the sake of this
example, we could declare our {\tt reg} module having a {\tt clk} port with the auto-bind
marker on it (Listing~\ref{lst1:reg_definition}).

\begin{lstlisting}[
  caption={\tt reg} definition,
  label=lst1:reg_definition,
]
module reg
#(
  parameter WIDTH = 1,
) (
  auto input logic clk,
  input logic [WIDTH-1:0] in,
  output logic [WIDTH-1:0] out
);
\end{lstlisting}

\subsection{Conditional statements}
The last large remaining use-case for {\tt @always} blocks is the use of conditional
assignments in the form of {\tt if} and the variants of {\tt case} statements.

The actual circuitry described by these constructs is almost always some type of
multiplexer. These can already be described as continuous assignments, using the
{\tt ?:} operator.

Verilog offers a more convenient way of using conditional statements in continuous
assignments: the {\tt always\_comb} block. We propose to elevate the syntax used
in those blocks to the module level, as shown in Listing~\ref{lst1:conditionals}.

\begin{lstlisting}[
  caption=conditionals,
  label=lst1:conditionals,
]
module collatz_step
(
  auto input logic clk,
  input logic [63:0] in,
  output logic [63:0] out
);
  if (in[0] == 0) begin
    out <= reg(in >> 1)
  end else begin
    out <= reg(in << 1 + in + 1)
  end
endmodule
\end{lstlisting}

!!!!!!!!!!!!!!!!!!!!!!!!!!!!!!!!!!!!!!!!!!!!!!!!!!!!!!!!!!!!!!!!!!!!!!!!!!!!!!!!!!!!!!!!!!!!!!!!

THIS IS NOT WORKING. THERE REALLY ISN'T AN ARGUMENT WHY MOST OF THIS COULDN'T BE DONE IN VERILOG.
VERIBLE MIGHT EVEN BE THE RIGHT VEHICLE TO CREATE A PRE-PROCESSOR THAT TURNS NEW-STYLE VERILOG
INTO TRADITIONAL ONE. WHERE IT WOULD START BECOMING PAINFUL IS ALL THE META-PROGRAMMING STUFF.
WHILE EVEN THAT COULD BE DONE IN VERILOG, I ASSUME, IT WOULD BE TOO MUCH SURGERY. THAT'S WHERE
A PYTHON-BASED IMPLEMENTATION SHINES, BUT THAT'S A MUCH MORE DIFFICULT ARGUMENT TO MAKE.

ANOTHER THING I'VE REALIZED IS THIS: NO ONE WILL INCORPORATE ANY OF THIS INTO THEIR ASIC WORKFLOW.
THOSE LONG SPAGHETTI-CODE TCL SCRIPTS WILL NOT BE MODIFIED TO DO AN EXTRA PRE-PROCESSING STEP,
VERIBLE OR PYTHON. THESE ARE NOT MAKEFILE PROJECTS AND EVEN IF THEY WERE, NO ONE WOULD CHANGE
THEM.

IN OTHER WORDS, A PROJECT LIKE THIS WILL **HAVE TO** START AT THE BOTTOM, AND CONVINCE FPGA
USERS ON A SMALL SCALE FIRST. THEN, MAYBE, IT CAN MIGRATE UP THE FOOD-CHAIN, BUT IT'S HIGHLY
QUESTIONABLE. REALLY THE ONLY REASON TO ADOPT SOMETHING LIKE THIS IS THE LIBRARY SUPPORT.

THAT, OR IF YOU CAN'T STAND VERILOG.

!!!!!!!!!!!!!!!!!!!!!!!!!!!!!!!!!!!!!!!!!!!!!!!!!!!!!!!!!!!!!!!!!!!!!!!!!!!!!!!!!!!!!!!!!!!!!!!!

Other statements, such as {\tt case} could be handled the same way.

Implementing all of these features within an existing toolset would not be
terribly difficult, but can only be done by the maintainer of the toolset.
Worse, these features would only be useful if most vendors implemented them.
Prior to that, there wouldn't be much user appetite for using the features
and without demand there would be no incentive for vendors to expend resources
implementing these changes. A classic Catch-22.

An alternative could be to create a Verilog pre-processor that takes
a new-style Verilog source-code and outputs a toolset-compatible version of it,
translating the new constructs into old ones. The problem here is that there are
very few open-source Verilog parser implementations, and those are very far
behind

https://github.com/chipsalliance/verible - Probably the most promising. Comes from Google.
http://iverilog.icarus.com/ - Verilog (and some SV) simulator
https://www.veripool.org/verilator/ - Verilog (or SV) to C++ compiler.
https://pypi.org/project/pyverilog/ - limited to no SV support, seems abandoned, single-dev driven


% An example of a floating figure using the graphicx package.
% Note that \label must occur AFTER (or within) \caption.
% For figures, \caption should occur after the \includegraphics.
% Note that IEEEtran v1.7 and later has special internal code that
% is designed to preserve the operation of \label within \caption
% even when the captionsoff option is in effect. However, because
% of issues like this, it may be the safest practice to put all your
% \label just after \caption rather than within \caption{}.
%
% Reminder: the "draftcls" or "draftclsnofoot", not "draft", class
% option should be used if it is desired that the figures are to be
% displayed while in draft mode.
%
%\begin{figure}[!t]
%\centering
%\includegraphics[width=2.5in]{myfigure}
% where an .eps filename suffix will be assumed under latex,
% and a .pdf suffix will be assumed for pdflatex; or what has been declared
% via \DeclareGraphicsExtensions.
%\caption{Simulation results for the network.}
%\label{fig_sim}
%\end{figure}

% Note that the IEEE typically puts floats only at the top, even when this
% results in a large percentage of a column being occupied by floats.


% An example of a double column floating figure using two subfigures.
% (The subfig.sty package must be loaded for this to work.)
% The subfigure \label commands are set within each subfloat command,
% and the \label for the overall figure must come after \caption.
% \hfil is used as a separator to get equal spacing.
% Watch out that the combined width of all the subfigures on a
% line do not exceed the text width or a line break will occur.
%
%\begin{figure*}[!t]
%\centering
%\subfloat[Case I]{\includegraphics[width=2.5in]{box}%
%\label{fig_first_case}}
%\hfil
%\subfloat[Case II]{\includegraphics[width=2.5in]{box}%
%\label{fig_second_case}}
%\caption{Simulation results for the network.}
%\label{fig_sim}
%\end{figure*}
%
% Note that often IEEE papers with subfigures do not employ subfigure
% captions (using the optional argument to \subfloat[]), but instead will
% reference/describe all of them (a), (b), etc., within the main caption.
% Be aware that for subfig.sty to generate the (a), (b), etc., subfigure
% labels, the optional argument to \subfloat must be present. If a
% subcaption is not desired, just leave its contents blank,
% e.g., \subfloat[].


% An example of a floating table. Note that, for IEEE style tables, the
% \caption command should come BEFORE the table and, given that table
% captions serve much like titles, are usually capitalized except for words
% such as a, an, and, as, at, but, by, for, in, nor, of, on, or, the, to
% and up, which are usually not capitalized unless they are the first or
% last word of the caption. Table text will default to \footnotesize as
% the IEEE normally uses this smaller font for tables.
% The \label must come after \caption as always.
%
%\begin{table}[!t]
%% increase table row spacing, adjust to taste
%\renewcommand{\arraystretch}{1.3}
% if using array.sty, it might be a good idea to tweak the value of
% \extrarowheight as needed to properly center the text within the cells
%\caption{An Example of a Table}
%\label{table_example}
%\centering
%% Some packages, such as MDW tools, offer better commands for making tables
%% than the plain LaTeX2e tabular which is used here.
%\begin{tabular}{|c||c|}
%\hline
%One & Two\\
%\hline
%Three & Four\\
%\hline
%\end{tabular}
%\end{table}


% Note that the IEEE does not put floats in the very first column
% - or typically anywhere on the first page for that matter. Also,
% in-text middle ("here") positioning is typically not used, but it
% is allowed and encouraged for Computer Society conferences (but
% not Computer Society journals). Most IEEE journals/conferences use
% top floats exclusively.
% Note that, LaTeX2e, unlike IEEE journals/conferences, places
% footnotes above bottom floats. This can be corrected via the
% \fnbelowfloat command of the stfloats package.




\section{Conclusion}
The conclusion goes here.




% conference papers do not normally have an appendix



% use section* for acknowledgment
\ifCLASSOPTIONcompsoc
  % The Computer Society usually uses the plural form
  \section*{Acknowledgments}
\else
  % regular IEEE prefers the singular form
  \section*{Acknowledgment}
\fi


The authors would like to thank...





% trigger a \newpage just before the given reference
% number - used to balance the columns on the last page
% adjust value as needed - may need to be readjusted if
% the document is modified later
%\IEEEtriggeratref{8}
% The "triggered" command can be changed if desired:
%\IEEEtriggercmd{\enlargethispage{-5in}}

% references section

% can use a bibliography generated by BibTeX as a .bbl file
% BibTeX documentation can be easily obtained at:
% http://mirror.ctan.org/biblio/bibtex/contrib/doc/
% The IEEEtran BibTeX style support page is at:
% http://www.michaelshell.org/tex/ieeetran/bibtex/
%\bibliographystyle{IEEEtran}
% argument is your BibTeX string definitions and bibliography database(s)
%\bibliography{IEEEabrv,../bib/paper}
%
% <OR> manually copy in the resultant .bbl file
% set second argument of \begin to the number of references
% (used to reserve space for the reference number labels box)
\begin{thebibliography}{1}

\bibitem{IEEEhowto:kopka}
H.~Kopka and P.~W. Daly, \emph{A Guide to \LaTeX}, 3rd~ed.\hskip 1em plus
  0.5em minus 0.4em\relax Harlow, England: Addison-Wesley, 1999.

\end{thebibliography}




% that's all folks
\end{document}


